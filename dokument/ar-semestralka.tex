\documentclass[12pt,a4paper]{article}
\usepackage[utf8]{inputenc}
\usepackage[czech]{babel}
\usepackage[T1]{fontenc}
\usepackage{amsmath}
\usepackage{amsfonts}
\usepackage{amssymb}
\usepackage{graphicx}
\usepackage{mathtools}
\usepackage{fancyhdr}
\usepackage{epstopdf}
\usepackage{float}
\usepackage{gensymb}
\usepackage[final]{pdfpages}
\title{Automatické řízení \\
	Semestrální práce}
\usepackage[top=25mm, left=35mm, right=25mm, bottom=25mm]{geometry}
\author{Miroslav Bulka, Jan Cibulka}
\date{81.121.1025}	
\setlength{\parskip}{10pt}
\begin{document}
\maketitle

\begin{figure}[h]
	\centering
	\includegraphics[width=9cm]{obrazky/fav.png}
\end{figure}
\clearpage
\newpage
\section{Zadání}
\section{Řešení - Model neurčitosti}
\subsection{První úkol}
Výpočet ustáleného stavu.
\subsection{Druhý úkol}
Linearizace ve dvou pracovních bodech.
\subsubsection{Konstantní průtoky - mění se hladina}
\label{sec:2A}
\subsubsection{Konstantní hladina - mění se průtoky}
\label{sec:2B}
\subsection{Třetí úkol}
Přenos systému, nquist asi, oba pracovní body, neurčitost.
\subsubsection{Určení numerické neurčitosti}
\subsubsection{Definovaní modelu s pertrubacemi, nominální model, váhová funkce}
\subsection{Čtvrtý úkol}
Porovnání neurčitostí z \ref{sec:2A} a \ref{sec:2B}.
\section{Řešení - Návrh regulátoru}
\subsection{První úkol}
Parametry PI regulatoru. Nejsem si jistej jestli tady jde o subukoly nebo jenom podminky pro jeden ukol.
\subsubsection{Vnitřní stabilita uzavřené smyčky (Nquistovo kritérium)}
\subsubsection{Robustnost ve stabilitě}
\subsubsection{Podmínka útlumu komplementrání citlivostní funkce}
\subsubsection{Energie šumu omezená.}
\subsection{Druhý úkol}
Harmonické poruchy.
\subsection{Třetí úkol}
Maximální kolísání hladiny.
\subsection{Čtvrtý úkol}
Určení hodnoty nějakých signálů.


\end{document}      
